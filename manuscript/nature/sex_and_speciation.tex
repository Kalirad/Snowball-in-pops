%% Template for a preprint Letter or Article for submission
%% to the journal Nature.
%% Written by Peter Czoschke, 26 February 2004
%%

\documentclass{nature}

%% make sure you have the nature.cls and naturemag.bst files where
%% LaTeX can find them

\bibliographystyle{naturemag}

\title{Does Sex hinder speciation? Perspective from a computational model of RNA folding}

%% Notice placement of commas and superscripts and use of &
%% in the author list

\author{Ata Kalirad$^{1}$, Bauthor$^2$ \& Ricardo B. R. Azevedo$^2$}


\begin{document}

\maketitle

\begin{affiliations}
 \item School of Biological Sciences, Institute for Research in Fundamental Sciences (IPM), Tehran
 \item Department of Biology and Biochemistry, University of Houston, Texas 77204-5001
 \end{affiliations}

\begin{abstract}
The genetics of speciation has been generally ascribed to the negative epistasis between, otherwise benign, alleles at different loci in the hybrids \cite{Maheshwari2011}. In this view, neutral or adaptive mutations arise and fix in different lineages independently, and such accumulation makes it more likely for mutations from different lineages to be incompatible with each other \cite{Orr1995}. Assuming that populations are monomorphic during there evolution, as in strong selection weak mutation (SSWM) regime,  is a valid approach to simulate evolution as a series of beneficial mutations going to fixation \cite{Sniegowski2010}, but it can be problematic in understanding speciation between lineages, since it neglects the possible effect of population dynamic on the emergence of incompatibilities.
\end{abstract}

The genetics of speciation has been generally ascribed to the negative epistasis between, otherwise benign, alleles at different loci in the hybrids (reviewed in \cite{Maheshwari2011}). In this view, neutral or adaptive mutations arise and fix in different lineages independently, and such accumulation makes it more likely for mutations from different lineages to be incompatible with each other \cite{Orr1995}. Assuming that populations are monomorphic during there evolution, as in strong selection weak mutation (SSWM) regime,  is a valid approach to simulate evolution as a series of beneficial mutations going to fixation \cite{Sniegowski2010}, but it can be problematic in understanding speciation between lineages, since it neglects the possible effect of population dynamic on the emergence of incompatibilities. Recent studies have presented us with an inconvenient and yet intriguing reality: incompatibilities are segregating within species \cite{Seidel2008, Corbett-Detig2013, Hou2014, Chae2014}. 

\begin{figure}
\caption{Each figure legend should begin with a brief title for
the whole figure and continue with a short description of each
panel and the symbols used. For contributions with methods
sections, legends should not contain any details of methods, or
exceed 100 words (fewer than 500 words in total for the whole
paper). In contributions without methods sections, legends should
be fewer than 300 words (800 words or fewer in total for the whole
paper).}
\end{figure}

\section*{Another Section}

Sections can only be used in Articles.  Contributions should be
organized in the sequence: title, text, methods, references,
Supplementary Information line (if any), acknowledgements,
interest declaration, corresponding author line, tables, figure
legends.

Spelling must be British English (Oxford English Dictionary)

In addition, a cover letter needs to be written with the
following:
\begin{enumerate}
 \item A 100 word or less summary indicating on scientific grounds
why the paper should be considered for a wide-ranging journal like
\textsl{Nature} instead of a more narrowly focussed journal.
 \item A 100 word or less summary aimed at a non-scientific audience,
written at the level of a national newspaper.  It may be used for
\textsl{Nature}'s press release or other general publicity.
 \item The cover letter should state clearly what is included as the
submission, including number of figures, supporting manuscripts
and any Supplementary Information (specifying number of items and
format).
 \item The cover letter should also state the number of
words of text in the paper; the number of figures and parts of
figures (for example, 4 figures, comprising 16 separate panels in
total); a rough estimate of the desired final size of figures in
terms of number of pages; and a full current postal address,
telephone and fax numbers, and current e-mail address.
\end{enumerate}

See \textsl{Nature}'s website
(\texttt{http://www.nature.com/nature/submit/gta/index.html}) for
complete submission guidelines.

\begin{methods}
Put methods in here.  If you are going to subsection it, use
\verb|\subsection| commands.  Methods section should be less than
800 words and if it is less than 200 words, it can be incorporated
into the main text.

\subsection{Method subsection.}

Here is a description of a specific method used.  Note that the
subsection heading ends with a full stop (period) and that the
command is \verb|\subsection{}| not \verb|\subsection*{}|.

\end{methods}

%% Put the bibliography here, most people will use BiBTeX in
%% which case the environment below should be replaced with
%% the \bibliography{} command.

%%\begin{thebibliography}{1}
%%\bibitem{dummy} Articles are restricted to 50 references, Letters
%%to 30.
%%\bibitem{dummyb} No compound references -- only one source per
%%reference.
%%\end{thebibliography}

\bibliography{refs}


%% Here is the endmatter stuff: Supplementary Info, etc.
%% Use \item's to separate, default label is "Acknowledgements"

\begin{addendum}
 \item Put acknowledgements here.
 \item[Competing Interests] The authors declare that they have no
competing financial interests.
 \item[Correspondence] Correspondence and requests for materials
should be addressed to A.B.C.~(email: myaddress@nowhere.edu).
\end{addendum}



%%
%% TABLES
%%
%% If there are any tables, put them here.
%%

\end{document}
